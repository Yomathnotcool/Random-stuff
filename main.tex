\documentclass[12pt,a4paper,english]{article}
\usepackage{mathtools}
\usepackage{mathabx} 
\usepackage[a4paper]{geometry}
\usepackage[utf8]{inputenc}
\usepackage[OT2,T1]{fontenc}
\usepackage{tcolorbox}
\usepackage[keeplastbox]{flushend}
\usepackage{color}
\usepackage{tikz-cd}
\usepackage{appendix}
\usepackage{babel}
\usepackage{dsfont}
\usepackage{amsmath}
\usepackage{amssymb}
\usepackage{amsthm}
\usepackage{stmaryrd}
\usepackage{color}
\usepackage{array}
\usepackage{hyperref}
\usepackage{graphicx}
\usepackage{mathtools}
\usepackage{natbib}
\usepackage[bb=boondox]{mathalfa}
\geometry{top=3cm,bottom=3cm,left=2.5cm,right=2.5cm}
\setlength\parindent{0pt}
\renewcommand{\baselinestretch}{1.3}

\newcommand\restr[2]{{% we make the whole thing an ordinary symbol
  \left.\kern-\nulldelimiterspace % automatically resize the bar with \right
  #1 % the function
  \vphantom{\big|} % pretend it's a little taller at normal size
  \right|_{#2} % this is the delimiter
  }}
  
% definition of the "structure"
\theoremstyle{plain}
\newtheorem{thm}{Theorem}[section]
\newtheorem{lem}[thm]{Lemma}
\newtheorem{prop}[thm]{Proposition}
\newtheorem{coro}[thm]{Corollary}
\newtheorem{claim}{Claim}


\theoremstyle{definition}
\newtheorem{conj}{Conjecture}
\newtheorem{defi}{Definition}
\newtheorem*{example}{Example}
\newtheorem{exercise}{\textbf{\textcolor{red}{Exercise}}}
\newtheorem{step}{Step}

\theoremstyle{remark}

\newtheorem*{rem}{Remark}

% define new control sequence
\newcommand{\homo}{\mathbf{Hom}}
\newcommand{\Max}{\mathbf{Max}}
\newcommand{\spec}{\mathbf{Spec}}
\newcommand{\spm}{\mathbf{Spec}_{max}}
\newcommand{\Frac}{\mathbf{Frac}}
\newcommand{\tr}{\mathrm{tr}}
\newcommand{\codim}{\mathrm{codim}}
\newcommand{\dif}{\text{d}}
\newcommand{\jac}{\textbf{Jac}}
\newcommand{\der}{\textbf{Der}}
\newcommand{\rank}{\text{rank}}
\newcommand{\sym}{\textbf{Sym}}
\title{Random Stuff}
\date{\today}
\author{Deng Zhiyuan\footnote{Email:\ \href{mailto:dengzymathnt@outlook.com}{dengzymathnt@outlook.com}}}


\begin{document}
\maketitle
\newpage
In linear algebra and matrix theory, the axis of a linear transformation $T$ is the set of all vectors that are fixed by $T$, up to scalar multiplication. In other words, it is the subspace of the vector space that is mapped onto itself by the transformation $T$.

More precisely, the axis of $T$ is the set $V(T)$ of all vectors $v$ such that $T(v) = \lambda v$ for some scalar $\lambda$. Note that $V(T)$ is a subspace of the vector space on which $T$ acts.

If $T$ is a linear transformation represented by a matrix $A$ with respect to a basis for the vector space, then $V(T)$ is the nullspace of the matrix $A - \lambda I$, where $I$ is the identity matrix and $\lambda$ is the scalar such that $T(v) = \lambda v$ for some non-zero eigenvector $v$.

It is worth noting that the axis of $T$ is also known as the eigenspace of $T$.

Sure, here is an outline of the steps to perform a Möbius transformation from the upper half-plane to the unit disk:

Start with a function of the form $f(z) = \frac{az + b}{cz + d}$, where $a,b,c,$ and $d$ are complex numbers and $ad-bc\neq 0$. This function is a Möbius transformation.

Choose $a,b,c,$ and $d$ so that $f(i) = 0$ and $f(\infty) = 1$. We can choose $c = 0$ and $d = 1$ to simplify the calculation. Then, we have $f(i) = \frac{ai+b}{i} = b - ai$, so we need to choose $b = 0$ and $a$ to be a real number. Let's choose $a = -1$.

Substitute the chosen values of $a,b,c,$ and $d$ into the formula for $f(z)$ to obtain $f(z) = \frac{-z}{z + 1}$.

Use the transformation $g(z) = \frac{z-i}{z+i}$ to map the upper half-plane to the unit disk. This transformation maps $i$ to $0$ and $\infty$ to $1$. Applying this transformation to $f(z)$, we get $g(f(z)) = \frac{(z-i)/(z+i)}{(-z-i)/(z+i)} = \frac{i-z}{i+z}$.

Therefore, the Möbius transformation that maps the upper half-plane to the unit disk is $h(z) = \frac{i-z}{i+z}$, which is the same as $g(f(z))$.

\end{document}
