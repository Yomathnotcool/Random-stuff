\documentclass[12pt,a4paper,english]{article}
\usepackage{mathtools}
\usepackage{mathabx} 
\usepackage[a4paper]{geometry}
\usepackage[utf8]{inputenc}
\usepackage[OT2,T1]{fontenc}
\usepackage{tcolorbox}
\usepackage[keeplastbox]{flushend}
\usepackage{color}
\usepackage{tikz-cd}
\usepackage{appendix}
\usepackage{babel}
\usepackage{dsfont}
\usepackage{amsmath}
\usepackage{amssymb}
\usepackage{amsthm}
\usepackage{stmaryrd}
\usepackage{color}
\usepackage{array}
\usepackage{hyperref}
\usepackage{graphicx}
\usepackage{mathtools}
\usepackage{natbib}
\usepackage[bb=boondox]{mathalfa}
\geometry{top=3cm,bottom=3cm,left=2.5cm,right=2.5cm}
\setlength\parindent{0pt}
\renewcommand{\baselinestretch}{1.3}

\newcommand\restr[2]{{% we make the whole thing an ordinary symbol
  \left.\kern-\nulldelimiterspace % automatically resize the bar with \right
  #1 % the function
  \vphantom{\big|} % pretend it's a little taller at normal size
  \right|_{#2} % this is the delimiter
  }}
  
% definition of the "structure"
\theoremstyle{plain}
\newtheorem{thm}{Theorem}[section]
\newtheorem{lem}[thm]{Lemma}
\newtheorem{prop}[thm]{Proposition}
\newtheorem{coro}[thm]{Corollary}
\newtheorem{claim}{Claim}


\theoremstyle{definition}
\newtheorem{conj}{Conjecture}
\newtheorem{defi}{Definition}
\newtheorem*{example}{Example}
\newtheorem{exercise}{\textbf{\textcolor{red}{Exercise}}}
\newtheorem{step}{Step}

\theoremstyle{remark}

\newtheorem*{rem}{Remark}

% define new control sequence
\newcommand{\homo}{\mathbf{Hom}}
\newcommand{\Max}{\mathbf{Max}}
\newcommand{\spec}{\mathbf{Spec}}
\newcommand{\spm}{\mathbf{Spec}_{max}}
\newcommand{\Frac}{\mathbf{Frac}}
\newcommand{\tr}{\mathrm{tr}}
\newcommand{\codim}{\mathrm{codim}}
\newcommand{\dif}{\text{d}}
\newcommand{\jac}{\textbf{Jac}}
\newcommand{\der}{\textbf{Der}}
\newcommand{\rank}{\text{rank}}
\newcommand{\sym}{\textbf{Sym}}
\title{Random Stuff}
\date{\today}
\author{Deng Zhiyuan\footnote{Email:\ \href{mailto:dengzymathnt@outlook.com}{dengzymathnt@outlook.com}}}


\begin{document}
\maketitle
\newpage
In linear algebra and matrix theory, the axis of a linear transformation $T$ is the set of all vectors that are fixed by $T$, up to scalar multiplication. In other words, it is the subspace of the vector space that is mapped onto itself by the transformation $T$.

More precisely, the axis of $T$ is the set $V(T)$ of all vectors $v$ such that $T(v) = \lambda v$ for some scalar $\lambda$. Note that $V(T)$ is a subspace of the vector space on which $T$ acts.

If $T$ is a linear transformation represented by a matrix $A$ with respect to a basis for the vector space, then $V(T)$ is the nullspace of the matrix $A - \lambda I$, where $I$ is the identity matrix and $\lambda$ is the scalar such that $T(v) = \lambda v$ for some non-zero eigenvector $v$.

It is worth noting that the axis of $T$ is also known as the eigenspace of $T$.

Sure, here is an outline of the steps to perform a Möbius transformation from the upper half-plane to the unit disk:

Start with a function of the form $f(z) = \frac{az + b}{cz + d}$, where $a,b,c,$ and $d$ are complex numbers and $ad-bc\neq 0$. This function is a Möbius transformation.

Choose $a,b,c,$ and $d$ so that $f(i) = 0$ and $f(\infty) = 1$. We can choose $c = 0$ and $d = 1$ to simplify the calculation. Then, we have $f(i) = \frac{ai+b}{i} = b - ai$, so we need to choose $b = 0$ and $a$ to be a real number. Let's choose $a = -1$.

Substitute the chosen values of $a,b,c,$ and $d$ into the formula for $f(z)$ to obtain $f(z) = \frac{-z}{z + 1}$.

Use the transformation $g(z) = \frac{z-i}{z+i}$ to map the upper half-plane to the unit disk. This transformation maps $i$ to $0$ and $\infty$ to $1$. Applying this transformation to $f(z)$, we get $g(f(z)) = \frac{(z-i)/(z+i)}{(-z-i)/(z+i)} = \frac{i-z}{i+z}$.

Therefore, the Möbius transformation that maps the upper half-plane to the unit disk is $h(z) = \frac{i-z}{i+z}$, which is the same as $g(f(z))$.




We can define the group homomorphism as follows:

\begin{align} \phi: \Gamma &\rightarrow \mathbb{R}/2\pi\mathbb{Z} \ \gamma &\mapsto \frac{1}{2\pi}\text{Log}(a_\gamma) \end{align}

where $a_\gamma$ is the real part of any non-trivial fixed point of $\gamma$.

This homomorphism is well-defined because any Fuchsian group $\Gamma$ can be represented as a discrete subgroup of the group of Möbius transformations, and hence any $\gamma \in \Gamma$ has a non-trivial fixed point in the extended complex plane $\mathbb{C}\cup{\infty}$. Moreover, the choice of fixed point $a_\gamma$ does not affect the value of $\phi(\gamma)$ modulo $2\pi\mathbb{Z}$.

To show that $\phi$ is a group homomorphism, we need to verify that $\phi(\gamma_1\gamma_2) = \phi(\gamma_1) + \phi(\gamma_2)$ for all $\gamma_1,\gamma_2 \in \Gamma$. Let $a_1$ and $a_2$ be non-trivial fixed points of $\gamma_1$ and $\gamma_2$, respectively. Then, we have:

\begin{align} \phi(\gamma_1\gamma_2) &= \frac{1}{2\pi}\text{Log}(a_{\gamma_1\gamma_2}) \ &= \frac{1}{2\pi}\text{Log}(a_{\gamma_1}a_{\gamma_2}) \ &= \frac{1}{2\pi}\text{Log}(a_{\gamma_1}) + \frac{1}{2\pi}\text{Log}(a_{\gamma_2}) \ &= \phi(\gamma_1) + \phi(\gamma_2) \end{align}

Therefore, $\phi$ is a group homomorphism from $\Gamma$ to $\mathbb{R}/2\pi\mathbb{Z}$.



We can define a group homomorphism $\rho: \Gamma \to \mathbb{R}/2\pi\mathbb{Z}$ by sending each element $\gamma \in \Gamma$ to its translation length $L(\gamma) \in \mathbb{R}$.

Recall that the translation length of an element $\gamma \in \Gamma$ is defined as $$L(\gamma) = \lim_{n \to \infty} \frac{d(\gamma^n x, x)}{n}$$ where $x \in \mathbb{H}$ is any fixed point and $d$ is the hyperbolic distance.

It can be shown that $L(\gamma)$ is well-defined and lies in the interval $(0,\infty)$. Moreover, $L(\gamma_1\gamma_2) = L(\gamma_1) + L(\gamma_2)$ for any $\gamma_1, \gamma_2 \in \Gamma$. This implies that $\rho$ is a group homomorphism.

Now, we want to show that the image of $\rho$ lies in $\mathbb{R}/2\pi\mathbb{Z}$. Suppose $L(\gamma) \in \mathbb{R}$ is the translation length of some $\gamma \in \Gamma$. Then, for any $x \in \mathbb{H}$, we have $$d(\gamma x, x) = L(\gamma)$$ Let $g \in \Gamma$ be any element such that $gx \neq x$. Then, using the triangle inequality, we have \begin{align} d(g\gamma x, gx) &\leq d(g\gamma x, \gamma x) + d(\gamma x, x) + d(x, gx) \ &= d(\gamma x, x) + d(x, gx) \ &= L(\gamma) + d(x, gx) \end{align} Taking the limit as $n \to \infty$ in the definition of $L(\gamma)$, we get $$L(\gamma) = \lim_{n \to \infty} \frac{d(\gamma^n x, x)}{n} \leq \lim_{n \to \infty} \frac{d(g\gamma^n x, gx)}{n} + \lim_{n \to \infty} \frac{d(x, gx)}{n} = L(\gamma)$$ This implies that $L(\gamma) = d(gx, x)$ for any $g \in \Gamma$ such that $gx \neq x$.

Now, let $h \in \Gamma$ be an element such that $hx \neq x$. Then, we have \begin{align} 2\pi \cdot \rho(h) &= 2\pi \cdot L(h) \ &= 2\pi \cdot d(hx, x) \ &= \log(e^{2\pi d(hx,x)}) \\
&= \log(e^{i2\pi d(hx,x)}) \ &= [\text{Argument of the complex number } hx - x] \pmod{2\pi} \ &\in \mathbb{R}/2\pi\mathbb{Z} \end{align} Thus, $\rho(h)$ lies in $\mathbb{R}/2\pi\mathbb{Z}$ for any $h \in \Gamma$.

Therefore, $\rho$ is a group homomorphism from $\Gamma$ to $\mathbb{R}/2\pi\mathbb{Z}$.
\end{document}
